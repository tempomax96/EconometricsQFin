\documentclass{article}\usepackage[]{graphicx}\usepackage[]{color}
% maxwidth is the original width if it is less than linewidth
% otherwise use linewidth (to make sure the graphics do not exceed the margin)
\makeatletter
\def\maxwidth{ %
  \ifdim\Gin@nat@width>\linewidth
    \linewidth
  \else
    \Gin@nat@width
  \fi
}
\makeatother

\definecolor{fgcolor}{rgb}{0, 0, 0}
\newcommand{\hlnum}[1]{\textcolor[rgb]{0.502,0,0.502}{\textbf{#1}}}%
\newcommand{\hlstr}[1]{\textcolor[rgb]{0.651,0.522,0}{#1}}%
\newcommand{\hlcom}[1]{\textcolor[rgb]{1,0.502,0}{#1}}%
\newcommand{\hlopt}[1]{\textcolor[rgb]{1,0,0.502}{\textbf{#1}}}%
\newcommand{\hlstd}[1]{\textcolor[rgb]{0,0,0}{#1}}%
\newcommand{\hlkwa}[1]{\textcolor[rgb]{0.733,0.475,0.467}{\textbf{#1}}}%
\newcommand{\hlkwb}[1]{\textcolor[rgb]{0.502,0.502,0.753}{\textbf{#1}}}%
\newcommand{\hlkwc}[1]{\textcolor[rgb]{0,0.502,0.753}{#1}}%
\newcommand{\hlkwd}[1]{\textcolor[rgb]{0,0.267,0.4}{#1}}%
\let\hlipl\hlkwb

\usepackage{framed}
\makeatletter
\newenvironment{kframe}{%
 \def\at@end@of@kframe{}%
 \ifinner\ifhmode%
  \def\at@end@of@kframe{\end{minipage}}%
  \begin{minipage}{\columnwidth}%
 \fi\fi%
 \def\FrameCommand##1{\hskip\@totalleftmargin \hskip-\fboxsep
 \colorbox{shadecolor}{##1}\hskip-\fboxsep
     % There is no \\@totalrightmargin, so:
     \hskip-\linewidth \hskip-\@totalleftmargin \hskip\columnwidth}%
 \MakeFramed {\advance\hsize-\width
   \@totalleftmargin\z@ \linewidth\hsize
   \@setminipage}}%
 {\par\unskip\endMakeFramed%
 \at@end@of@kframe}
\makeatother

\definecolor{shadecolor}{rgb}{.97, .97, .97}
\definecolor{messagecolor}{rgb}{0, 0, 0}
\definecolor{warningcolor}{rgb}{1, 0, 1}
\definecolor{errorcolor}{rgb}{1, 0, 0}
\newenvironment{knitrout}{}{} % an empty environment to be redefined in TeX

\usepackage{alltt}


% packages:
\usepackage{amsmath , amssymb , amsthm}
\usepackage{a4wide}
\usepackage{amsfonts}
\usepackage{mathtools}
\usepackage{graphicx, verbatim}  


\newcommand{\E}{\mathbb{E}}
\newcommand{\Eps}{\mathcal{E}}
\newcommand{\G}{\Gamma}
\newcommand{\1}{\mathbb{1}}
\newcommand{\V}{\mathbb{V}}


\title{QFin - Econometrics 2 - Homework 1}
\author{Group 2: Fojkar, Hrustsova, Kuttner, Merkinger, Sünderhauf}
\IfFileExists{upquote.sty}{\usepackage{upquote}}{}
\begin{document}
\setlength\parindent{0pt}
\maketitle



\subsection*{Part A}
\textbf{Estimate a simple regression of real investment (realinvs) on a constant and the nominal interest rate (90 day treasury bill rate; tbilrate).} \\\\


\begin{knitrout}
\definecolor{shadecolor}{rgb}{0.933, 0.933, 0.933}\color{fgcolor}\begin{kframe}
\begin{alltt}
\hlstd{fit_a} \hlkwb{<-} \hlkwd{lm}\hlstd{(}\hlkwc{data} \hlstd{= df,} \hlkwc{formula} \hlstd{= realinvs} \hlopt{~} \hlstd{tbilrate)}
\hlkwd{summary}\hlstd{(fit_a)}
\end{alltt}
\begin{verbatim}
## 
## Call:
## lm(formula = realinvs ~ tbilrate, data = df)
## 
## Residuals:
##    Min     1Q Median     3Q    Max 
## -427.2 -239.9 -188.1  152.9 1122.5 
## 
## Coefficients:
##             Estimate Std. Error t value Pr(>|t|)    
## (Intercept)  397.194     53.897   7.369  4.3e-12 ***
## tbilrate      48.781      9.058   5.385  2.0e-07 ***
## ---
## Signif. codes:  0 '***' 0.001 '**' 0.01 '*' 0.05 '.' 0.1 ' ' 1
## 
## Residual standard error: 367.2 on 202 degrees of freedom
## Multiple R-squared:  0.1255,	Adjusted R-squared:  0.1212 
## F-statistic:    29 on 1 and 202 DF,  p-value: 2.001e-07
\end{verbatim}
\end{kframe}
\end{knitrout}


\subsection*{Part B}
\textbf{Estimate a multiple regression of real investment (realinvs) on a constant, real GDP (realgdp), the nominal interest rate (90 day treasury bill rate; tbilrate) and the inflation rate (infl).}\\\\

\begin{knitrout}
\definecolor{shadecolor}{rgb}{0.933, 0.933, 0.933}\color{fgcolor}\begin{kframe}
\begin{alltt}
\hlstd{fit_b} \hlkwb{<-} \hlkwd{lm}\hlstd{(}\hlkwc{data} \hlstd{= df,} \hlkwc{formula} \hlstd{= realinvs} \hlopt{~} \hlstd{realgdp} \hlopt{+} \hlstd{tbilrate} \hlopt{+} \hlstd{infl)}
\hlkwd{summary}\hlstd{(fit_b)}
\end{alltt}
\begin{verbatim}
## 
## Call:
## lm(formula = realinvs ~ realgdp + tbilrate + infl, data = df)
## 
## Residuals:
##     Min      1Q  Median      3Q     Max 
## -232.38  -39.88    6.61   32.37  289.45 
## 
## Coefficients:
##               Estimate Std. Error t value Pr(>|t|)    
## (Intercept) -1.532e+02  1.591e+01  -9.633  < 2e-16 ***
## realgdp      1.856e-01  3.192e-03  58.158  < 2e-16 ***
## tbilrate    -8.923e+00  2.912e+00  -3.064  0.00249 ** 
## infl         1.234e+00  2.227e+00   0.554  0.58002    
## ---
## Signif. codes:  0 '***' 0.001 '**' 0.01 '*' 0.05 '.' 0.1 ' ' 1
## 
## Residual standard error: 85.32 on 199 degrees of freedom
##   (1 observation deleted due to missingness)
## Multiple R-squared:  0.9532,	Adjusted R-squared:  0.9525 
## F-statistic:  1350 on 3 and 199 DF,  p-value: < 2.2e-16
\end{verbatim}
\begin{alltt}
\hlcom{# 'infl' has no significant influence i.e. estimated coefficient is not }
\hlcom{# significantly different from zero}
\end{alltt}
\end{kframe}
\end{knitrout}


\subsection*{Part C}
\textbf{Interpret the coefficient/slope of tbilrate from (a).}\\\\

\subsection*{Part D}
\textbf{Consider the coefficient of tbilrate obtained from (b). Interpret that coefficient and try to 'explain' (to the extent possible) what determines the difference obtained from the coefficient in (a).}



\subsection*{Part E}
\textbf{Use the results from (b) and check the implications 1-4 from section 1.1.2.}

\end{document}
